%% ---------------------------------
%% | Information about the thesis  |
%% ---------------------------------

\newcommand{\myname}{Nils Braun}
\newcommand{\mytitle}{Title \\[0.75cm]
		\huge{Title}}
\newcommand{\myinstitute}{Institut für experimentelle Kernphysik (IEKP)}
\newcommand{\myinstituteen}{Institut of Experimental Nuclear Physics (IEKP)}

\newcommand{\reviewerone}{Prof. Dr. Michael Feindt}
\newcommand{\reviewertwo}{Prof. Dr. Ulrich Husemann}

\newcommand{\timestart}{November 2014}
\newcommand{\timeend}{November 2015}
\newcommand{\submissiontime}{DATUM}

%% -------------------------------
%% |  Information for PDF file   |
%% -------------------------------
\hypersetup{
	pdfauthor={Nils Braun},
	pdftitle={},
	pdfsubject={},
	pdfkeywords={}
}


%% --------------------------------
%% | Settings for word separation |
%% --------------------------------
% Help for separation:
% In german package the following hints are additionally available:
% "- = Additional separation
% "| = Suppress ligation and possible separation (e.g. Schaf"|fell)
% "~ = Hyphenation without separation (e.g. bergauf und "~ab)
% "= = Hyphenation with separation before and after
% "" = Separation without a hyphenation (e.g. und/""oder)

% Describe separation hints here:
\hyphenation{
} 


\LetLtxMacro{\oldtodo}{\todo}
\renewcommand{\todo}[2][]{\oldtodo[inline, #1]{#2}}

% Listen
\newcounter{RRR}
\newenvironment{Rlist}{
  \begin{list}{(\Roman{RRR})}{
      \usecounter{RRR}\setlength{\leftmargin}{8mm}\setlength{\labelsep}{2mm}
    }
}{\end{list}}

\newcounter{rrr}
\newenvironment{rlist}{
  \begin{list}{(\roman{rrr})}{
      \usecounter{rrr}\setlength{\leftmargin}{8mm}\setlength{\labelsep}{2mm}
    }
}{\end{list}}

\newcounter{zzz}
\newenvironment{zlist}{
  \begin{list}{(\arabic{zzz})}{
      \usecounter{zzz}\setlength{\leftmargin}{8mm}\setlength{\labelsep}{2mm}
    }
}{\end{list}}

\newcounter{abc}
\newenvironment{alist}{
  \begin{list}{(\alph{abc})}{
      \usecounter{abc}\setlength{\leftmargin}{8mm}\setlength{\labelsep}{2mm}
    }
}{\end{list}}

% Mathematische Symbole
\newcommand{\nM}{\mathbb}
\newcommand{\nR}{\mathbb{R}}
\newcommand{\nN}{\mathbb{N}}
\newcommand{\nZ}{\mathbb{Z}}
\newcommand{\nQ}{\mathbb{Q}}
\newcommand{\nC}{\mathbb{C}}
\newcommand{\nK}{\mathbb{K}}
\newcommand{\nF}{\mathbb{F}}
\newcommand{\nullel}{\mathcal{O}}
\newcommand{\einsel}{\mathds{1}}

\newcommand{\summe}[2]{\sum\limits_{#1}^{#2}}

\newcommand{\coss}[1]{\cos\left( #1 \right)}
\newcommand{\sinn}[1]{\sin\left( #1 \right)}
\newcommand{\psumme}{\sum\limits_{n=0}^{\infty}}


\newcommand{\limesn}{\lim\limits_{n\to\infty}}
\newcommand{\limesx}{\lim\limits_{x\to\infty}}
\newcommand{\limesp}[1]{\lim\limits_{#1\to\infty}}
\newcommand{\limespfeil}[1]{\xrightarrow[#1\to\infty]{}}
\newcommand{\limesto}[2]{\lim\limits_{#1 \to #2}}
\newcommand{\limes}[1]{\lim\limits_{#1}}
\newcommand{\limespfeilto}[1]{\xrightarrow[#1]{}}
\newcommand{\limesw}[1]{\xrightarrow[#1]{W}}





\newcommand{\dd}[2]{\frac{\mathrm d#1}{\mathrm d#2}}
\newcommand{\pp}[2]{\frac{\partial#1}{\partial#2}}
\newcommand{\ddx}{\frac{\mathrm d}{\mathrm dx}}
\newcommand{\ddt}[1]{\frac{\mathrm d #1}{\mathrm dt}}
\newcommand{\ddn}[2]{\frac{\mathrm{d}^{#2}}{\mathrm{d}#1^{#2}}}
\newcommand{\ddxn}[1]{\frac{\mathrm{d}^{#1}}{\mathrm{d} x^{#1}}}
\newcommand{\bint}[2]{\int\limits_{#1}^{#2}}
\newcommand{\aint}[1]{\int\limits_{#1}^{}}
\newcommand{\intd}{\ \mathrm d}
\renewcommand{\div}{\text{div }}
\newcommand{\rot}{\text{rot }}

\newcommand{\diag}[1]{\mathrm{diag}\left(#1\right)}

\newcommand{\LL}{\mathcal{L}}

\newcommand{\grad}{\text{grad} \ }

\newcommand{\basf}{\texttt{basf2}}

% for double arrows a la chef
% adapt line thickness and line width, if needed
%\tikzstyle{vecArrow} = [thick, ->, >=stealth,
%   double distance=7pt, shorten >= 5.5pt,
%   postaction = {draw,line width=7pt, white,shorten >= 4.5pt}]
\tikzstyle{vecArrow} = [thick, ->, >=stealth]

\tikzstyle{module} = [rectangle, draw, fill=kit-green50, 
      text width=8em, text centered, rounded corners]
\tikzstyle{module-background} = [rectangle, draw, fill=kit-green15, rounded corners]
      
\tikzstyle{cloud} = [draw, ellipse,fill=red!20, node distance=3cm,
      minimum height=2em]
\tikzstyle{tracks} = [draw, -latex']
\tikzstyle{hits} = [draw=red!60, -latex', fill=red!60]

\tikzset{ header node/.style = {
    text depth    = +0pt,
    fill          = white,
    draw},
  header/.style = {%
    inner ysep = +1.5em,
    append after command = {
      \pgfextra{\let\TikZlastnode\tikzlastnode}
      node [header node] (header-\TikZlastnode) at (\TikZlastnode.north) {#1}
    }
  }
}

\lstdefinestyle{customP}{
  belowcaptionskip=1\baselineskip,
  breaklines=false,
  %frame=L,
  xleftmargin=\parindent,
  language=Python,
  showstringspaces=false,
  basicstyle=\footnotesize\ttfamily,
  keywordstyle=\bfseries\color{kit-green70},
  commentstyle=\itshape\color{kit-blue70},
  identifierstyle=\color{black},
  stringstyle=\color{kit-orange100},
}

\lstdefinestyle{customC}{
  belowcaptionskip=1\baselineskip,
  breaklines=false,
  %frame=L,
  xleftmargin=\parindent,
  language=C++,
  showstringspaces=false,
  basicstyle=\footnotesize\ttfamily,
  keywordstyle=\bfseries\color{kit-green70},
  commentstyle=\itshape\color{kit-blue70},
  identifierstyle=\color{black},
  stringstyle=\color{kit-orange100},
}

\definecolor{light-gray}{gray}{0.98}

% the space reserved between for the ``In'' numbers and the code
% after http://tex.stackexchange.com/questions/223465/ipython-notebook-cells-with-listings
\newlength\inwd
\setlength\inwd{1.6cm}

\newcounter{ipythcntr}

\newtcblisting{inputipynb}[1][\theipythcntr]{
  enlarge left by=\inwd,
  width=\linewidth-\inwd,
  enhanced,
  boxrule=0.2pt,
  colback=light-gray,
  listing only,
  listing options={
    style=customP
  },
  top=-6pt,
  bottom=-6pt,
  left=4pt,
  overlay={
    \node[
      anchor=north east,
      text width=\inwd,
      font=\footnotesize\ttfamily\color{kit-blue100},
      inner ysep=2mm,
      inner xsep=0pt,
      outer sep=0pt
      ] 
      at (frame.north west)
      {\stepcounter{ipythcntr}In [#1]:};
  }
}

\newtcblisting{outputipynb}[1][\theipythcntr]{
  enlarge left by=\inwd,
  width=\linewidth-\inwd,
  enhanced,
  boxrule=0pt,
  frame hidden,
  colback=white,
  listing only,
  listing options={
    basicstyle=\footnotesize\ttfamily,
  },
  top=-6pt,
  bottom=-6pt,
  left=4pt,
  overlay={
    \node[
      anchor=north east,
      text width=\inwd,
      font=\footnotesize\ttfamily\color{kit-blue100},
      inner ysep=2mm,
      inner xsep=0pt,
      outer sep=0pt
      ] 
      at (frame.north west)
      {Out [#1]:};
  }
}
