\chapter*{Summary}

As described in the introduction, the track finding plays a crucial role for the physics analyses performed on the data taken by the planned Belle~II experiment. The need for very high finding efficiencies and also high purities make a high demand on the tracking algorithms which should additionally be fast and highly configurable.

Within the scope of this thesis, the algorithms for the tracking in the central drift chamber (CDC) detector where analyzed, refined or newly developed. The stereo hit finder and the combiner for segments from the local and tracks from the global finder as well as tools for improving the track quality where firstly build during this work. The performed changes could improve the number of found tracks and hits, the rate of fitted tracks and also the computing time significantly.

Together with the improvements on the tracking figures of merit some parts of the Legendre track finding software was reworked and adapted to modern programming styles and a common code basis. As this thesis was the first to evaluate the track finding package in the CDC, some errors in the software could be corrected and the validation features could be improved. Additionally, the modularity of and the connectivity between the tracking modules for the CDC detector was increased, which made it easier to try out new possibilities and approaches for combining the track finders. Also, this thesis was the first to summarize the characteristics of the CDC track finding algorithms to give additional hints where there is room for improvements. Some of these improvements have already been implemented into the software.

The last step in the tracking procedure -- the track fitting -- still imposes some sever problems to the track finding in the CDC detector which have to be solved before the start of the experiment. After that, the influence of the various particle properties on the track finder has to be analyzed to be able to optimize the algorithms to their later use case. As all shown figures of merit rely on the correctness of the event and the background Monte Carlo simulation, these have to be checked and possibly corrected in the early phases of the experiment.

Another part of this thesis consisted of a first analysis of incorporating the $\mathrm dE/\mathrm d x$ value as another measurement into the trajectory fit for low momentum tracks in the vertex detectors to improve the momentum resolution. An estimator for calculating the momentum with the energy loss for each hit individually was generated and different possibilities to do so where compared. The estimation was then inserted into the fitting procedure with configurable parameters. The momentum resolution and the number of successful fits for momenta below $\unit[80]{MeV}$ could be increased drastically. Further developments and analysis to improve the error estimation and the transverse position resolution should be performed. Therefore, the newly created track class and the additional modules are easy to extend and can handle further changes.

% Also, some new functionalities as the \texttt{ipython\_handler} or the \texttt{root\_pandas} module in the software framework and the \texttt{RecoTrack} data structure in the tracking package were implemented.

The refined tracking algorithms -- especially for the CDC detector -- which were partly developed in the thesis are expected to improve the figures of merit of the tracking in the Belle~II detector. They have therefore a strong influence on all analyses which will be performed on the taken data. The designed combination of the two track finders could not only improve the number of correctly added hits of the found tracks but makes it also possible to find new tracks out of the remaining segments. Therefore, a significant improvement over the reference implementation is expected in the figures of merit and also in the computing time which opens a wide window of new possibilities for refined physics analyses.