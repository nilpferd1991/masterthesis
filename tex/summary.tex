\chapter*{Summary}

\todo{Summary trackfinding cdc}

The second part of this thesis consisted of a first analysis of incorperating the $\mathrm dE/\mathrm d x$ value as another measurement into the helix fit for low momentum tracks to improve the momentum resolution. The reolution -- especially for momenta in the range of $\unit[50]{MeV}$ -- could be increased drastically. Further developments and analysis to improve the error estimation and the transversal position resolution must be performed. The newly created \texttt{RecoTrack} interface with the new \texttt{MeasurementCreatorModule} is due to the used factory design pattern more easy to extend.

Together with the improvements on the tracking figures of merit some parts of the Legendre track finding software was reworked and adapted to modern programming styles and a common code basis. As this thesis was the first one evaluating the track finding package in the CDC, some errors in the package could be corrected and the validation features could be improved. Also some new functionalities as the \texttt{ipython\_handler} or the \texttt{root\_pandas} module in the software framework and the \texttt{RecoTrack} data structure in the tracking package were implemented.

The last step in the tracking procedure -- the track fitting -- still imposes some sever problems to the track finding in the CDC detector which have to be solved before the start of the experiment. After that, the influence of the various particle properties to the track finder has to be analyzed to be able to optimize the algorithms to their later use case. As all shown figures of merit rely on the correctness of the event and the background simulation ...